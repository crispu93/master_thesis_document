%
% File: chap02.tex
%
\let\textcircled=\pgftextcircled
\chapter{Experimental Results}
\label{Chapter4}
The number of partitions was set to three 

For the dataset used to train and test the algorithm "The Graph Partitioning Archive"~\cite{archive}

For the purposes of this research, the graphs contained in "The Graph Partitioning Archive" where categorized in one of the following categories according to their vertex cardinality: 
\begin{itemize}
    \item Tiny graphs: those with less than $10,000$ nodes,
    \item Small graphs: those with node size between $45,000$ and $75,000$ and,
    \item Medium graphs: those with at least $99,000$ nodes but no more than $500,000$.
\end{itemize}

Ask Luis: is there a word for graphs with size more than big

\begin{table}
\centering
\begin{tabular}{ |p{1.75cm}||cc|  }
\hline
\multicolumn{3}{|c|}{\textbf{Computation Graphs}} \\
\hline
\hline
\textbf{Name} & \textbf{Nodes} & \textbf{Edges} \\
\hline
add20 & 2395 & 7462  \\
data & 2851 & 15093  \\
3elt & 4720 & 13722  \\
uk & 4824 & 6837  \\
add32 & 4960 & 9462  \\
bcsstk33 & 8738 & 291583  \\
whitaker3 & 9800 & 28989  \\
crack & 10240 & 30380  \\
\hline
fe\_body & 45087 & 163734  \\
t60k & 60005 & 89440  \\
wing & 62032 & 121544  \\
finan512 & 74752 & 261120 \\
\hline
fe\_rotor & 99617 & 662431  \\
598a & 110971 & 741934  \\
m14b & 214765 & 1679018	 \\
auto & 448695 & 3314611  \\
\hline
\end{tabular}
\caption{\label{tab:results}Summary of the graphs characteristics. Taken from "The Graph Partitioning Archive" ~\cite{archive}}
\end{table}

In table 

\begin{table}
\centering
\begin{tabular}{ |p{1.75cm}||cc|cc|cc|  }
%\hline
%\multicolumn{3}{|c|}{\textbf{Results}} \\
\hline
\hline
\textbf{Graphs} & \multicolumn{2}{c}{\textbf{METIS}} & \multicolumn{2}{c}{\textbf{GAP}} & \multicolumn{2}{c|}{\textbf{Modified GAP}} \\
\hline
\hline
Name & Edge Cut & Balancedness & Edge Cut & Balancedness & Edge Cut & Balancedness \\
\hline
add20 & 0 & 0 & 0 & 0 & 0 & 0  \\
data & 0 & 0 & 0 & 0 & 0 & 0  \\
3elt & 0 & 0 & 0 & 0 & 0 & 0  \\
uk & 0 & 0 & 0 & 0 & 0 & 0  \\
add32 & 0 & 0 & 0 & 0 & 0 & 0  \\
bcsstk33 & 0 & 0 & 0 & 0 & 0 & 0  \\
whitaker3 & 0 & 0 & 0 & 0 & 0 & 0  \\
crack & 0 & 0 & 0 & 0 & 0 & 0  \\
\hline
fe\_body & 0 & 0 & 0 & 0 & 0 & 0  \\
t60k & 0 & 0 & 0 & 0 & 0 & 0  \\
wing & 0 & 0 & 0 & 0 & 0 & 0  \\
finan512 & 0 & 0 & 0 & 0 & 0 & 0  \\
\hline
fe\_rotor & 0 & 0 & 0 & 0 & 0 & 0  \\
598a & 0 & 0 & 0 & 0 & 0 & 0  \\
m14b & 0 & 0 & 0 & 0 & 0 & 0  \\
auto & 0 & 0 & 0 & 0 & 0 & 0  \\
\hline
\end{tabular}
\caption{\label{tab:comp_graphs}Comparison of the results obtained by different algorithms in the computation graphs}
\end{table}