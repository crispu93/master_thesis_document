
\chapter{Conclusion}
\label{Conclusion}
Las conclusiones y el trabajo a futuro inicia aqu\'i...

Future work
extend to different problem statements, such as balanced graph partitioning with tolerance or weighted graph partitioning.
Study the real impact on choosing the way to extract the features and try different algorithms
look for new ways the algorithm relies only on topological graph's structure
To generalize the algorithm and code to construct graphs from the features
Explore new techniques for sampling
Build a generic graph construction
Training the algorithm with different graphs, look for useful training sets that allow to extend the recognition of different structural properties in graphs. For example, training the graph on only sparse or complete graphs
Explore improvements in the training of the algorithm, parallel tasks or mix between CPU and GPU
GraphSAGE still takes a lot of time to run on really large graphs
Extend the accepted formats 
Study another way to extract features
\section{Contributions}
while it remains an open problem and it still very unexplored, the problem of feature generation...
\section{Recommendations and future work}

\clearpage