

\chapter*{Resumen}
\addcontentsline{toc}{chapter}{Resumen}

El diseño de algoritmos para resolver problemas de Optimización Combinatoria es una tarea desafiante debido a la naturaleza de intractabilidad que poseen los mismos. Uno de los problemas fundamentales, y también uno de los más estudiados en el área, es el problema del \textit{Particionado de Grafos}. En años recientes, se han desarrollado distintos algoritmos basados en Deep Learning para lidiar con éste y otros problemas de Optimización Combinatoria, los cuales han mostrado resultados satisfactorios. Uno de las algoritmos más famosos, entre los antes mencionados, es \textit{Generalizable Approximate Partitioning (GAP) framework}. El objetivo de este trabajo es presentar una extensión de este \textit{framework} que funcione para grafos mas generales, i.e. grafos sin atributos, y generalice bien para grafos grandes.  

Una de las primeras observaciones hechas en esta investigación es que debido al hecho que el framework GAP usa una arquitectura basada en \textit{Graph Convolutional Networks}, éste tiende a desempeñarse lentamente en grafos grandes. Por lo tanto, se llevaron a cabo algunas modificaciones significantes que mantienen las ventajas que ofrecen las Graph Convolutional Networks y al mismo tiempo reducen el tiempo de cómputo sin perder la potencia del framework original.

Además, se observó que el framework GAP funciona únicamente con grafos con características en sus nodos como consecuencia de la arquitectura seleccionada por sus autores. Esta dependencia fue elmininada a través de un enfoque basado en caminatas aleatorias para generar las características de los nodos. Esta se puede considerar como una modificación importante porque el algoritmo ahora depende únicamente de la estructura del grafo lo cuál es suficiente para el problema del Particionado de Grafos.

La última parte de esta investigación muestra cómo se adaptó una técnica de \textit{negative sampling} al algoritmo propuesto para acelerar y hacer más efectivo el proceso de encontrar particiones balanceadas sin sacrificar la calidad de las particiones antes mencionadas. El algoritmo propuesto mostró resultados comparables con algoritmos ampliamente usados para el particionado de grafos como lo es METIS.

%\clearpage